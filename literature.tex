\chapter{Image based localisation}
\label{chap:visualnavigate}
%
%
\vspace{-10.0em}
\begin{abstract}
The previous chapter mentioned some navigation tools 
which assist the navigation. This chapter moves to a 
next level and reviews works which 
offer visual navigation for robotics or 
visually impaired people. The localisation based 
on different technologies is reviewed to understand the 
importance of vision based localisation. 
Finally, we present some good research works 
which use robust computer vision techniques to 
perform scene matching in indoor and outdoor 
environments. 
\end{abstract}



\section{Visual Navigation}
The navigation based on vision (images or videos) 
is referred as "walk through problem". 
The vision sensor seems to be the best choice 
for navigation if we consider its nice features 
like low price, low power, non contact and 
high potential information contents. 
However, its really hard to extract useful 
and reliable information from the image feeds 
to base mobility decisions.
The visual navigation focuses on guidance with the 
goal of automatically reproducing the tasks performed
by humans without any visual impairment. The important tasks 
are to detect the current location, plan the route 
and then guide the blind person safely 
while avoiding the possible obstacles on the way. 
 Francisco \emph{et al.} provides a very good survey 
on visual navigation for mobile robots 
and mention key research contributions 
which are made from nineties till 2008 (~\citet{bonin-font08}).
The visual navigation systems are
divided into two main categories:
% (1)
%map-based systems which need some 
%representation of the environment in 
%the form of a map before the start 
%of navigation, and (2) mapless systems 
%which do not need any map aof the environment. 
%In mapless systems, navigation is 
%determined by observing and extracting
%useful features in the environment 
%based on on board sensors.



\subsection{Map based systems}
\label{sec:mapbasedsystem}

Map based systems need 
some representation of the environment 
built before the start of navigation process. 
There are two systems mainly 
(1) systems which use an existing 
map of the environment 
usually referred as \emph{map-using}, 
and (2) systems that 
build the map of the environment themselves 
usually 
referred as \emph{mup-building}. 
The navigation starts only 
after the environment representation or a 
map is available.

%Map building and self-localization in the navigation environment 
%are two functionalities that most systems tend to incorporate. 
%In standard map-building approaches, 
%the localization in the environment can be computed
%by some other techniques while in pure 
%localization approaches, the map of the
%environment is presumably available. 
For a true autonomous visual navigation, the 
system needs to do automatic exploration, 
localisation and mapping of the environment 
via techniques known as Simultaneous localization and mapping (SLAM).
SLAM basically involves the simultaneous 
estimation of a map and the robot pose concurrently.
%Normally 
%systems based on SLAM solve each of the two 
%problems independently i.e. a robot can trivially 
%estimate its position given a complete map of its environment 
%and it can also construct a map from sensor data if i
%ts position is accurately known. 
Recently, cameras have been used as the sole sensor
to yield SLAM based on vision i.e. visual SLAM (~\citet{chen07, davison07,silveira08}).
A standard scheme to visual SLAM consists 
of first extracting a sufficiently
large set of features and robustly matching
them between successive frames. These corresponding features
are used for estimating the camera pose
and scene structure. In this section, 
we discuss some well known SLAM systems 
which make use of a map 
with a particular focus on indoor buildings. 
%with a focus on indoor buildings.

Davison presented MonoSLAM which 
is based on monocular vision and is one of the most 
successful schemes for a single camera (~\citet{davison03}). 
Landmark positions and camera pose 
are estimated and refined directly from 
observations in the image using an 
Extended Kalman Filter (EKF).
The camera can be positioned accurately 
for extended periods provided that sufficient landmarks 
are tracked. However, the positioning is limited to 
small scale environments with 100 
landmarks in total. Clement \emph{et al.} 
(~\citet{clemente07}) 
extended MonoSLAM to deal with 
larger environments by using 
the submaps framework proposed by (~\citet{estrada05}).
MonoSlam is used on small scale and new map is started 
once the number of landmarks grows beyond
some limit. The local submaps are combined into 
an accurate global map by opimising transformations 
between submaps. The system is able to close 
250m loop with an accurate global map.



Sim \emph{et al.} presented a system which maps 
a large and complex environment in real time using 
a pair of stereo cameras (~\citet{sim06}). 
The main contribution of the work 
is a fully automatic mapping system which operates 
on-line and consistently produces accurate maps of 
large scale environments. The system uses a hybrid approach 
consisting of 3D landmark extraction based on 
SIFT features (~\citet{lowe04}) for 
image localisation and occupancy grid 
to obtain a map for a safe navigation. 
The occupancy grid represents
the environment as an evenly spaced 
field of binary random variables each 
representing the presence of an obstacle 
at that location in the environment.
The occupancy grid offers a reliable 
spatial representation of the 
world for a safe navigation. 
The proposed system is deployed on 
a robot and it does well in a laboratory 
environment with of two rooms during the testing. 


%A topological map is a graph based representation 
%of the environment and is often used for navigation. 
%Each node corresponds to a place in the environment 
%and is associated with actions 
%such as turning, crossing a door etc.
Topological maps are graph based representation 
of the environment and are 
parsimonious representation because they are 
simple and easy to scale. Winters \emph{et al.} (~\citet{winters99} 
proposed a system for robot which 
uses omnidirectional camera images to generate a 
topological map of the indoor structured 
environment during a training phase.
The links in the graph are sequence of images 
belonging to a place (e.g. corridor) and nodes 
represent specific places associated with actions 
such as turning, crossing a door etc. 
The image set is represented by %compressed by %compressed by Principal Component Analysis to obtain 
a low dimensional eigenspace and robot determines
its position by projecting the current image into eigenspace 
during movement. After position estimation, 
the robot uses ground plane images to extract 
corridor guidelines to control its trajectory. 
The system does well in corridors of 
Institute for Systems and Robotics, Portugal (ISR) 
\footnote{\url{http://welcome.isr.ist.utl.pt/home/}}.


Kidono \emph{at al.} developed a system 
which needs training from a human to build a map (~\citet{kidono02}). 
As user guides a mobile robot to a
destination by remote control, 
the robot constructs a 3D model on line incrementally 
frame by frame from stereo images. 
For autonomous movement, the robot 
utilizes the map and past experience on observation 
to safely navigate from the starting point 
to the goal point.  The robot is navigated through an 
indoor room with different obstacles and did well.
%. During this movement, the robot observes the surrounding
%environment to make a map. Once the map is generated, the robot computes and follows
%the shortest path to the destination autonomously. To
%With the map, 
%the robot can can repeat the sane route from 
%starting point to the goal point via tracking the features 
%and finding the closest safe path. 



%Anther important map building robot applications are 
%the museum guiding robots which need to be autonomous 
%to recognize people, guide them through different environments 
%and also avoid obstacles. 
Thrun \emph{et al.} developed a museum guiding robot 
MINERVA\footnote{\url{http://www.cs.cmu.edu/~minerva/}} that uses laser scans, camera images and 
odometry readings to to build map of the environment for 
the navigation (~\citet{thrun99}). The robot is first navigated 
via joystick in the museum to generate 
occupancy and ceiling texture maps. The occupancy 
grid ensure safe navigation and texture maps are used 
for localisation in case of a huge crowd because 
frontal images do not offer sufficient information for 
feature based tracking. The robot remained operational for two weeks 
at Smithsonian’s National Museum of American History 
and successfully interacted with thousands of people.
Shen \emph{et al.} proposed ATLAS, a more advanced 
museum guiding robot (~\citet{shen06}). 
The laser data and images are used 
to match the map data of the environment for localisation. 
A visual appearance based algorithm is than used for normal
topological navigation. In visual appearance algorithms, 
the current image is compared with stored templates or models 
for the recognition. ATLAS is tested in the foyer of London County
Hall for 03 months. The robot can (1) detect the nearby visitors 
and interact with them via voice and a touching screen, (2)
uses a human face detection to approach to new visitors
and (3) can detect the battery level and
charge itself automatically. This makes ATLAS possible to
run continuously in a museum environment.


%The approaches discussed so far use global 
%description of the environment obtained either 
%automatically or by a human guided stage before the start of navigation. 
Some navigation systems support on line construction of a local 
occupancy grid which is the portion of 
the environment surrounding the robot.
This local information is used for a 
subsequent map construction frame by frame 
for on line safe navigation. 
Gartshore \emph{et al.}
presented a framework for map building 
based on online images captured by a single camera
 (~\citet{gartshore02}). 
%The work focuses on locating the 
%position of features in 
%the environment during the map building.
The main contribution of the work 
is combination of occupancy 
grid and visual data from a single camera.
The system uses feature detector 
algorithm to identify the object boundaries 
based on features in occupancy grids 
from the current frame. While positioning 
module computes robot position using 
odometry data and detected features. 
The color or gradient from 
features from past frames are used 
to further verify the object presence 
at a specific location. 
%The systems computes the 
%probabilities of finding the objects 
%at every location. The edge features are used to detect 
%object boundaries for the current frame.
%The positioning module of the system computes the 
%position of robot via odometry data and 
%with extracted features. The color or gradient 
%from edges from past frames help to verify the 
%object presence at a certain location. 
The experiments were conducted in indoor environment 
and robot was able to navigate safely while 
avoiding the obstacles. 

Botterril \emph{et al.} presented a BoWSLAM scheme 
which enables real-time navigation of robots 
in dynamic environments using a single camera alone 
(~\citet{botteril10}).
The robot can position itself in real-time while exploring 
a previously unknown environment. 
The system works by representing 
every frame as a Bag of Words (BoW). BoW 
representation is then used to select 
multiple nearby frames to compute relative positions 
and latest frame is positioned relative to each of these 
frames. A subset of these multiple position hypotheses 
with minimum gross errors is selected 
to accurately position the robot in a global
map. The proposed system allows robot navigation in 
challenging dynamic and self-similar environments.


\subsection{Mapless systems}
These systems refer to those techniques which do not need 
any knowledge of the environment but they navigate 
as they perceive the environment. These techniques 
often grab video frames to produce enough information 
about the unknown environment for a safe navigation. 

Optical flow is one of the technique commonly used 
for mapless systems. Taludker \emph{et al.} presented 
a novel optical flow based solution 
for robots to detect the presence of dynamic objects 
in the camera field of view during navigation (~\citet{talukder04}).
The algorithm assumes that moving objects cause discontinuity 
in optical flow orientation and changes its magnitude with respect 
to the background pixels orientation and magnitude.
The system is initially tested with single camera and 
later on with stereo cameras to extract the depth information.


Green \emph{et al.} presented a Closed Quarter Aerial Robot (CQAR) 
whose navigation is controlled 
by an insect inspired optical flow based system  (~\citet{green03}) . 
The robot can fly into buildings, take off and land 
based on optical flow computed from the images 
of the environment. The minimum flying speed of 
CQAR is 2 m/s and it needs to turn
%, the turning radius is about 2.5 m and it needs 
5 meters before to avoid an obstacle. 

 Zhou \emph{et al.} used appearance based strategy 
for mapless navigation  (~\citet{zhou03}). Such strategies 
has two phases (1) first in a pre-training phase,
the images or prominent features from the environment 
are observed and stored as model templates. 
These models are labeled with a certain location 
information, % or with an associated control steering command, 
and (2) in the navigation stage, the
robot matches the current image with the stored templates 
to recognize the environment and self-localize. 
Zhou \emph{et al.} 
utilized color, gradient, edge density and texture 
histograms to describe the appearance of pre-recorded 
indoor images. The navigation is performed by matching 
the multidimensional histogram of current image with 
the stored templates. The use of histograms save 
computation resources and are simple to use.
The main problems with appearance based strategies are
to identify the ways to represent the environment 
and perform on-line matching. 

The image qualitative characteristics and their interpretation 
are often used by visual techniques to navigate 
and avoid obstacles. The navigation systems 
based on qualitative information avoid as
much as possible the obstacles in the environment. 
%The are two types of reactive visual obstacle 
%avoidance systems (1) model-based systems, 
%which need pre-defined models of known objects, and 
%(2) sensor-based systems, which process 
%every online sensor information to determine the 
%free space and obstacles. 
Fasola \emph{et al.} proposed to use 
image color segmentation techniques 
to detect the possible objects from frames 
followed by classification of opponent robots 
via gray scale umage processing (~\citet{fasola06}). 
The integral images which were introduced for real time 
face detection have been used for a robust robot classification 
(~\citet{viola01}). The proposed system is developed
with the annual RoboCup Competition in mind 
where teams of Sony AIBO 4-Legged robots compete
in the game of soccer. From a set of 327 test
images, the system was able to achieve a 97\% 
classification accuracy yielding only one false positive.


The techniques for tracking moving elements 
in a video sequence have become robust enough and useful 
for navigation. Such techniques divide a tracking task into two sub-tasks
(1) motion detection, which refers to the identification 
of most likely region in next frame to find the feature, 
and (2) feature matching, by which the 
feature tracked is identified within the determined region (~\citet{trucco06}).
Such techniques do not handle the obstacle avoidance.
SIFT features (~\citet{lowe04}) are the popular features 
for the navigation so far. During the navigation, the 
SIFT features are observed from different viewpoints, angles, distances 
and with different illumination changes. The detected features serve as 
appropriate landmarks to be trace over time for navigation, 
global localisation and a robust vision based SLAM performance 
(~\citet{se01, se05}).

Saeedi used stereo vision in a novel navigation strategy applicable to unstructured
indoor/outdoor environments (~\citet{saeedi06}). 
The main emphasis of this work is to estimate the robot motion
independently from any prior scene or landmark knowledge. 
The system uses a new, faster and more
robust corner detector and detected features are 3D positioned.
The 3D positions of scene features and the robot 
are refined by a Kalman filtering over time. The results 
indicate a good tracking and localization performance covering 
outdoor environment of 5 cm by the system.

\section{Localisation}
\label{sec:localisation}
In the previous section, we have reviewed 
some key contributions in area of visual navigation 
for robotics. However, same techniques can be utilized 
for blind people by replacing the control signals 
with appropriate voice messages.
 
However, the focus of our work is the scene localisation 
in indoor environment only and not the navigation. 
Technologies other than vision have been used 
for indoor positioning, such as infrared,
ultrasonic etc. It is important to briefly review these techniques 
to understand the corresponding limitations to 
identify the benefits of vision based localisation. 
In this section, we briefly discuss indoor positioning based on 
different techniques for the analysis (~\citet{indoorsurvey11, gu09}). 


\subsection{Infrared}
\label{sec:infrared}
Infrared (IR) positioning technology needs infra-red 
emitters and measures the position of the object
according to the receiving time of infrared signals. 
Want \emph{et al.} developed Active Badge system, one
of the first successful IR based indoor positioning system at
AT\&T Cambridge in 1990s  (~\citet{want92, harter02, abs08}). The person needs 
to carry an active badge and system provides room level 
indoor localisation with active badges.
An active badge transmits a globally unique IR signal every 15
seconds. In each located place, one or more sensors are fixed 
and detect the IR signal sent by
an active badge. 
The position of an active badge can be
specified by the information from these sensors 
and forwards the location information
of the tracked active badges to a central server.
The price of active badges and networked sensors
are cheap but the cables connecting the 
sensors raise the cost of the
Active Badge system. 
%Moreover, the effective transmission 
%range of this positioning technology is
%only a few meters since the poor penetration of infrared.
%The system may not work well in glass structure buildings as 
%it is influenced by sunlight and fluorescent.

OPTOTRAK system is designed for positioning in 
congested shops and workspaces ~\cite{opo08}. The OPTOTRAK
uses a system of three cameras as a linear array to track 3-
D positions of numerous markers on an object. 
The markers mounted on different parts of a
tracked object emits IR light which is detected by the camera
of the system to estimate the location of them via triangulation 
technique. The system offers a high accuracy of 0.1 mm to 0.5 mm
with 95\% success probability. A disadvantage of OPTOTAK system
is the line-of-sight requirement between the objects and the
tracking system. However this can be partially solved 
by using a large number of IR markers at the expense of cost. 

IR based systems provide a precise position estimation 
%IR emitters are small, light-weight and the system 
%architectures are simple leading to a low 
%installation and maintenance cost. However, there 
%are still some disadvantages with these systems. 
However, IR signals have some limitations for sensing location, 
for example, interference from florescent light and sunlight. 
This problem can be solved by using optical and electronic filters 
but it raises the cost of the positioning system. 
Although, the IR emitters are cheap but whole system using camera array, 
transmitters, IR device and wire connectivity for each isolated 
place increases the cost of system.
connected via wires is expensive when used in large areas. 
%comparing to the coverage
%area. There should be a transmitter or receiver in every
%measured place such as a room equipped with at least one IR
%device to locate whether the target persons or devices are in
%the room or not. These transmitters or receivers fixed in each
%place are connected using special wire. In addition, when an
IR device taken by a person is covered by his/her clothes,
the system fails to work since the IR wave can not penetrate
opaque materials.

% developed ActiveBadge, 
%one of the best indoor positioning 
%working system so far. The system got refined 
%with time. Each user needs to carry small infrared
%marking equipment. The marking equipment sends a globally unique identification
%code every 15 seconds and infrared sensors 
%fixed in the building collect the data
%then transmit to a central server which accomplishes the positioning process. 
%ActiveBadge is low cost and easy to use, but it is 
%vulnerable to the impact of fluorescent and
%sunlight. Moreover, the effective transmission 
%range of this positioning technology is
%only a few meters since the poor penetration of infrared.
%The system may not work well in glass structure buildings 
%due to sunlight coming through the windows. The associated 
%cost is also higher. The largest single system is at Cambridge University Computer Laboratory, 
%where over 200 badges and 300 sensors are in daily use.

\subsection{Ultrasonic}
\label{sec:ultrasonic}
Using ultrasound signal is another way to measure the position. 
Ultrasound signals are used by bats to navigate
in the night which inspire people to design a similar navigating
system. Ultrasonic positioning technology uses mainly reflective distance method to
determine the location of the object. 

Active Bat system is a well known 
ultrasonic indoor positioning system researched by AT\&T Labs (~\citet{ultrasound04}).
It is a low-power, wireless indoor location system accurate up to 3 cm. 
It relies on multiple ultrasonic receivers embedded in the ceiling 
and measures time-of-flight to them. It works by finding 
the distance to minimum of three reference 
nodes and then using multilateration technique to find the exact position. 
The performance of the system is influenced
by the reflection and obstacles between tags and receivers,
which degrades the system accuracy. The scalability of the system 
is affected due to deployment of 
a large number of sensors on the ceiling. 
The receivers also need to be accurately placed,
which results in complex and costly installation.

Cricket system is another location system which offers 
efficient performance and low cost (~\cite{priyantha00,priyantha05}). 
The cricket system uses time of arrival measuring method
and triangulation location technique to locate a target. The
system includes ultrasound emitters as infrastructure
attached on the walls or ceilings at known positions, and a
receiver mounted on each object to be located.
The cost of the whole system is low and it can 
provide a position estimation accuracy of 10 cm.
The receivers in the cricket system consume more power 
and its power supply needs to be efficiently designed 
to avoid frequent charging.
The Sonitor ultrasound IPS is an indoor tracking
and positioning solution provided by Sonitor Technologies
Inc (~\citet{sonitor08}). The Sonitor system can locate and track people and
devices in real-time and offer very good room level accuracy. 

%Ultrasound positioning systems give a kind of inexpensive positioning
%solutions. Usually the ultrasound signals used to locate
%objects need to be combined with RF signals, which perform
%synchronization and coordination in the system. These ultrasound
%positioning systems increase the system coverage area.
%However, ultrasound-based positioning systems have lower
%measurement accuracy (several centimeters) than IR-based
%systems (several millimeters). These ultrasound positioning
%systems suffer from reflected ultrasound signals and other
%noise sources such as jangling metal objects, crisp packets,
%etc.
 Usually the ultrasound signals used to locate
objects need to be combined with RF signals, which perform
synchronization and coordination in the system. These ultrasound
positioning systems increase the system coverage area.
However, Ultrasonic positioning system requires 
large-scale layout with many of receiver hardware 
leading to a overall higher cost. 
The positioning result is very sensitive with the
environment and suffers from reflected signals, 
metal objects etc. 

%\subsection{ZigBee}
%\label{sec:zigbee}
%
%ZigBee is a short distance, low-rate wireless network technology 
%and achieves positioning with the coordination of 
%communications by thousands of tiny
%sensors. These sensors require very little energy to relay the data passing between the
%sensors leading to a very efficient communication. 
%The accuracy of such systems is limited and systems 
%become unstable with clusters of people. 
%The associated cost is also higher 
%due to use of large sensors.
%Wireless Dragon 
%positioning system is a representative 
%ZigBee positioning system developed by Chengdu 
%Wireless Dragon Communication Technology Co., Ltd. 

\section{RFID}
\label{sec:rfid}

Radio frequency (RF) technologies use frequency signals 
which can travel through walls and human bodies easier
therefore offering a larger coverage area and
less requirement of hardware. The RFID positioning
systems are commonly used in complex indoor environments
and offer cheap identification.


LANDMARCE positioning system is a RFID-based indoor
positioning system developed by Michigan State University and the Hong Kong
University of Science and Technology (~\citet{ni04}). 
The system uses fixed position reference tags and 
%which are placed in a fixed position 
and RFID readers to determine the nearest reference 
%the compare the signal strength from the target and 
%reference tags to determine 
tag from the target tag followed by locaiton estimation. 
%Location can be determined by the nearest reference and reference tags. 
Jin \emph{at al.} further improved LANDMARC with respect to system's energy consumption
and costs (~\citet{jin06}). 

WhereNet positioning system offers indoor and outdoor real-time positioning 
(~\citet{wherenet08}).
RFID technology is used to identify various located tags mounted 
on the target located objects, such as a device or a person.
The system uses differential time of arrival algorithm to 
calculate the precise locations of these tags.
The tags are powered by batteries which can last up to 7
years depending on the transmission rate of the tags. However,
the WhereNet offers an error range around 2 m to 3 m,
which is not very accurate in indoor situations. The system
is complex and the installation of these devices is
time consuming.

The advantage associated with RFID positioning system is light and small
tags that can be taken by people to be tracked. The RFID
system can uniquely identify equipment and persons tracked
in the system. However, the absolute positioning
techniques need numerous infrastructure components installed
and maintained in the working area of an RFID positioning
system which increases the overall cost.

%The disadvantage of such positioning systems is the RF signal
%influenced by the antenna, the role of proximity does not have the communications
%capabilities, positioning coverage is small, and not easily integrated into other
%systems. 

\subsection{WLAN}
\label{sec:wlan}

WLAN technology is very popular and 
is widely used in business districts, universities, airports etc. 
WLAN-based positioning systems reuse the
existing WLAN infrastructures in indoor environments, which
lower the cost of indoor positioning. In case of unavailability 
of WLAN infrastructure, the associated cost increases. 
This positioning technology uses the WLAN client such as 
laptop, PDA, phone etc to get the received signal strength (RSS) 
or signal to noise ratio (SNR) from wireless network interface card. 


RADAR positioning system was proposed
by a Microsoft research group as an indoor position tracking
system (~\citet{bahl00}). 
It uses signal strength and SNR with
the triangulation location technique. The RADAR system can
provide 2-D absolute position information and thereby enable
location-based applications for users.
The major advantages of RADAR system are that the
use of existing indoor WLAN infrastructures and requirement of
few base stations to perform location sensing. However, the
limitation is that the located object needs to be equipped with
WLAN technology which is difficult for some lightweight
and energy-limited devices. There is also no consideration
of privacy issues where a person 
using a device with WLAN interface may be tracked,
even he/she does not want any one know his/her location. In
addition, the RADAR system suffers from the limitations of
RSS positioning methodology.


The COMPASS system uses WLAN infrastructures 
and digital compasses to provide low cost and 
high accurate positioning services to locate
a user carrying a WLAN-enabled device (~\citet{king06}). 
The COMPASS system uses fingerprinting location technique
and a probabilistic positioning algorithm to determine the
location of a user. During position estimation, 
the user’s orientation is measured by a
digital compass to reduce the human body blocking influence.
For the tracking of a mobile user, the orientation impact
is highly addressed by the designers
of both RADAR and COMPASS system. As human body
contain more than 50\% water, which absorbs the 2.4 GHz
radio signal, the clocking effect of human body influences the
measurement accuracy. The COMPASS system achieves an 
accuracy of about 1.65 m compared to RADAR
system which has an error distance of 2.26 m in the same 
indoor place. However, the COMPASS system only considers tracking a
single user which reduces its scalability.

WLAN systems provide very cheaper 
indoor position estimation due to re-usability 
of existing infrastructures in indoor environments. 
WLAN technology is widely used
and integrated in various wireless devices such as PDAs,
laptops, mobile phones, etc. However, because of complex indoor
environments consisting of various influenced sources, 
the performance of the positioning systems are not very
accurate with an accuracy of several meters. And using the
stored information and fingerprinting technique in the location
estimations is complex and costly if the number of users of
the positioning system is increasing significantly.
The fingerprinting requires a costly and time consuming signal 
strength calibration at the kick off. Some recent solutions 
do not require calibration and an accuracy of 2-7 m is achievable 
which is good (~\cite{chintalapudi10}). 

\section{Audible Sound}
\label{sec:sound}
Audible sound is a possible technology for indoor positioning
as almost every mobile device has that ability.
% of emitting audible sound such as mobile phone, PDA, etc. 
%The audible sound-based positioning system can reuse these devices owned
%by the users for indoor positioning. Wearable tracked tags are
%no longer needed resulting in a low-cost system.
Beep was designed as a cheap positioning solution 
based on audible sound technology (~\citet{lopes06}).
Triangulation location technique is used in Beep with a standard
3-D multilateration algorithm based on time of arrival measured
by the sensors in Beep system. The 3-D location information
determined by Beep can be used by various practical applications.
%as proposed by Beep in the situations, such as office
%building, shopping center, etc.
%Figure 18 shows the architecture of the Beep positioning
%system. A roaming device is used as a tracked target in the
%system to send audible sound. Various acoustic sensors (Si)
%are pre-installed at fixed positions in the measuring area and
%connected to the central server through wireless connection
%(Ri). These sensors receive the audible sound transmitted from
%the tracked device and forward these data to the central server
%through WLAN. TOA technology and triangulation method
%are used to estimate the position of the device. Finally, the
%roaming device can get its position information from the
%central server via WLAN. The testing experiments were taken
%in a 20 m × 9 m room. The positioning system can achieve
The system gives an accuracy of 0.4 m with 90\% . 
In addition, the effect of sound noise and obstacles reduce the positioning
accuracy by 6-10\%. One of the benefits brought by the Beep system is that the
privacy of the users is considered by avoiding them being
tracked automatically. The users can stop their devices from
sending audible sound; if they do not want the system knows
their location.

Audible sound is an available service in various mobile devices used
in our daily lives. Thus the users can use their personal devices
in an audible sound positioning system to get their positions.
Because of properties of audible sound, using it for indoor
positioning has some limitations. The audible sound can be
interfered by the sound noises in the dynamic changing and
public indoor situations. Audible sound does not have high
penetration ability, so the scope of an infrastructure component
is within a single room. The associated cost will increase 
with more rooms. Transmitting audible sound is a kind
of noise to indoor environments, where people would not like
to hear audible sound made by the positioning services.

\section{Vision}
\label{sec:vision}
The main merit of vision basd system is 
a low price camera can covera large area. However,
these systems have some drawbacks. Firstly, the privacy
of people is not provided by the vision-based positioning.
Secondly, the system is not reliable in a dynamic changing
environment. However, the problem is often solved 
by using an incremental approahc in which data base 
is updated with new images. The visionbased
positioning is influenced by light such as turning on and off
a light. Our system targets to provide guidance to blind person 
who often visits the building during day time and can take multiple pictures 
to localisae the place. 


\section{Image based Localisation}
\label{sec:image}

Our work is based on scene localisation using computer vision 
which deals with the images. The basic idea is to match the 
current image with stored images and get the corresponding 
location information. 
Image based indoor location recognition
is an active research problem and has been widely studied 
by computer vision researchers. In this section, 
we discuss different good techniques which offer a robust scene 
recognition in different environments.


The visual Bag of Words (BoW) approach offers robust image 
matching and is used in many 
research works for location recognition. 
In Sivic and Zisserman (\cite{sivic03}), the idea of visual BoW
was introduced for the
first time for object retrieval. 
Given a query object, the
system is able to search 
and localize all the occurrences of 
that object in a video. The 
inverse document frequency ($idf$)
weighting scheme was used for image 
retrieval.

Nister and Stewenius (\cite{nister06}) used 
a scheme based on visual BoW
scalable to large image databases is presented.
Their scheme is based on hierarchical clustering 
and uses term frequency-inverse document frequency ($tfidf$)
weighting, which
is shown to be very effective 
for retrieval in large scale data sets.
The system is able to match an image 
robustly in about one second.

Tom \emph{et al.} presented a real-time approach 
for the detection of identical scenes (\cite{tom08}). 
The work is based on visual BoW which uses the
SURF features and hue information 
for a robust  image matching. The system matches 
a query image from a standard data set of 10200 images 
in about 0.036 seconds.


Filliat (\cite{davidf07}) used a two stage voting
scheme for indoor matching. 
SIFT, hue and texture features are used for visual
BoW and a room is recognized only
if a quality threshold is reached. 
The work is only tested for a small scale indoor 
environment but it has not been shown to work 
in office buildings which have similar 
color/texture schemes in many places.


In Kang \emph{et al.} (\cite{kang09}), visual BoW is used to perform matching
on a large scale indoor office environment. 
The top eight images most similar to the query image are 
retrieved via visual BoW. A potential localization
is suggested if there are a cluster of pre-recorded images less than 3 meters from each
other among the retrieved images. The proposed system
performs well in an office like indoor building.


Robertson and Cipolla (\cite{duncan04}) used homography and rectification 
for scene recognition in an outdoor environment.
In their work, camera's are assumed calibrated (or at least
approximately so), and database images are assumed rectified. Features
are identified using the Harris corner detector and a RANSAC based
algorithm for image registration is applied. The query image is
matched against each database image and the closest match is returned
as the location. 

In Zhang \emph{et al.}~\cite{zhang06}, the image localisation is performed in 
the urban environment. The system computes the 
GPS location of the novel query image 
from a database of city street scenes tagged with 
GPS locations. The two closest views relative to the query 
image are first retrieved from the database based on 
SIFT~\cite{lowe04} feature matching followed by 
the motion estimation. 
The location of the query 
image is finally estimated by the triangulation of GPS 
positions of the closest views. 

Skrypnyk \emph{et al.}~\cite{sky04} presented 
one of the first systems to estimate the pose 
of the query image with respect to 
the 3D model. A sparse 3D model 
from objects of interest is reconstructed using 
multi view geometry and the associated SIFT descriptors 
are stored in a kd-tree. A robust algorithm is then used 
to estimate the pose of a query object with respect to 
the 3D models via 2D-3D correspondences.

Sattler \emph{et al.}~\cite{torsten11} 
presented a direct 2D-3D matching framework 
based on visual vocabulary for image matching. 
A novel approach is proposed to find 2D-3D correspondences 
efficiently followed by the pose estimation. The system 
performs well on city scale large data sets.


In Irschara \emph{et al.}~\cite{irschara09}, 
a fast location recognition technique 
based on $sfm$  is presented. A minimal set of "synthetic" 
images which represent the 3D models of places is 
derived first to compress the 3D data. Given a query image, 
the system then uses the vocabulary tree to retrieve 
the similar images from this compressed data 
based on the pose estimation. 

Most research works focus on direct matching of
2D features (query) to 3D points (3D models) 
for localisation. However Li  \emph{et al.}~\cite{li10},
proposed an opposite approach based on 
3D-2D matching. The work uses a priority scheme 
to effectively compute the 2D-3D correspondences 
followed by the pose estimation. The proposed scheme works quite fast
on large scale data sets compared to the direct matching which is 
slow. 


Arth \emph{et al.}~\cite{arth2009}
used an efficient method to localise
the image on the mobile phone 
based on 3D models. The 
3D models of the places are 
reconstructed and part of it is loaded in the phone memory. 
The system extracts the features from the 
camera query image, loads the relevant part of 
the feature data base in the phone memory and performs 
the pose estimation to match the image. 
The experiments are conducted in indoor 
rooms.

Mulloni \emph{et al} \cite{mulloni09} proposed an indoor 
positioning system where the smart phone 
application sends the query photo to a server. 
The server uses SURF features \cite{bay06}, 
distance estimation and calibration 
to estimate the indoor position.
The system shows good performance 
on a long corridor. 

Hile \emph{et al} \cite{hile07} 
presented a smart phone application 
which sends the query image along 
with WIFI fingerprints to a server. 
The server computes feature correspondences 
with the floor plan to estimate the pose
and returns the position information to the 
user. The system does well with hallway images.


Kawaji \emph{et al.} \cite{kawaji10} proposed an image 
based indoor localization system using omnidirectional 
panoramic images. The system uses SIFT features \cite{lowe04} 
to match the query image with a database 
of mapped images by using nearest neighbor search 
based on local sensitive hashing. The system 
is shown to work in a Railway museum.


Ruf \emph{et al.} \cite{ruf08} integrated 
SIFT and SURF for image matching 
based on nearest neighbor search 
leading to precise position estimation inside 
a museum. However the proposed system 
produced acceptable accuracy at the expense of high computational
cost.

More recently, 
object detection and probabilistic semantics have been used in a
small scale indoor
environment to identify the place type~\cite{espinace10}. 
Their work should be applicable to indoor 
environments of any type/scale
but will perform slower as objects 
segmentation, objects classification 
and then use of semantics is usually slow.

The idea of visual BoW is used by different authors 
for robust object recognition in large scale data sets. 
Phibin \emph{et al.} used spatial information 
to verify the consistency of retrieved images 
to re-rank the retrieved results (~\citet{philbin07}).
The initially returned result list is re-ranked by estimating 
affine homographies between the query image and each
of the top-ranking results from the initial query. The score
used in re-ranking is computed from the number of verified 
inliers for each result.
Philbin \emph{et al.} later extended his work 
by exploring techniques to map visual region 
to a weighted set of words to include features
which get lost in the quantization stage (~\citet{philbin08}). 
The proposed method 
of visual assignment is found to perform better than
 state of the art techniques on different large data sets. 

Aly \emph{et al.} analysed the performance of visual 
BoW with two leading methods i.e. inverted file and 
mini-hash on four diverse real world large data sets (~\citet{aly09}). 
The effect of the different parameters
of these methods on the recognition performance
and the run time are analysed in the paper. The work 
basically targets object recognition.

Lazebnik \emph{at al.} presented the idea 
of a spatial pyramid with visual BoW (~\citet{lazebnik06}). 
The “spatial pyramid” is a simple and computationally efficient
extension of an orderless bag-of-features image representation 
to get improved performance in scene categorization tasks.
The proposed method works by repeatedly subdividing
an image and computing histograms of image features
over the resulting subregions. The system does well on 
standard datasets for image matching. 

Ji \emph{et al.} proposed a framework to perform 
landmark search via smart phone efficiently without the 
need to send query photo to the server (~\citet{ji12}).
When user starts the application, the GPS information is 
used to download the generalized visual vocabulary 
codebook on the phone. When user selects a query photo, 
local features are extracted from the phone 
followed by the generation of a compact Location Discriminative 
Vocabulary Code (LDVC) compact descriptor from the 
generalized vocabulary. The information normally in 10-50 bits 
is sent to the server for matching and results are returned 
to the phone. The proposed approach results in lower 
transmission rates and provide very good performance. 
The descriptor is sent to the server, 

\section{Conclusion}
\label{sec:conclusion}
In this chapter, we have discussed some key works in area of visual navigation. 
This was followed by comparison of different localization techniques.
Finally, we mention different research works which have used visual 
BoW to perform image matching in indoor and outdoor environments.
We also mention the limitations of these works to highlight the need 
of an indoor localisation system based on vision.

%\subsection{GPS-Based}
%GPS is not reliable in indoor due to 
%satellite signal attenuation. 
%SnapTrack\footnote{\url{http://www.qualcomm.com/}} 
%a Qualcomm company developed wireless assisted 
%GPS (A-GPS) to overcome the limitation of a GPS and 
%provide positioning with an accuracy of 5-50m in most environments.
%A-GPS uses the mobile phone network to 
%assist the GPS receiver in the mobile phone 
%to overcome the low signal level problems. 
%A-GPS uses a location server 
%with a reference GPS receiver that can 
%simultaneously detect the same satellites as wireless handset 
%with a partial GPS receiver. This helps the partial GPS receiver to 
%find weak GPS signals. The wireless handset collects measurements 
%from both GPS constellation and wireless mobile network 
%which are combined by location server to estimate the indoor position.
%The indoor position is then communicated to the handset.
%GPS receiver needs to be on every time and it should be installed at 
%a place where it can see all satellites. 
%%
%%Remazeilles \emph{et al.} (~\citet{remazeilles04}) used an appearance based navigation 
%%system. The system uses an image data base which is a set of views 
%%built off line representing the whole navigable environment.
%%Once navigation mission is defined, an image sequence corresponding to 
%%what robot should see during the motion is extracted from the data base.
%%The robot motion is the result of online detection and matching 
%%process between the models included in the sequence and 
%%perceived scenes.To navigate, the robot tracks recognizable previously cataloged 
%%features.
%%
%%Fraundorfer \emph{et al.} (~\citet{ fraundorfer})proposed vision based localisation and mapping 
%%for robot navigation. The environment is represented as a linked 
%%collection of way point images. The collection is built online while 
%%robot is exploring the environment.  Links are created between the 
%%sequential images and are inserted by image matching. 
%%An efficient image matching scheme allows real time mapping and global 
%%localsation. 
%%
%%Booji \emph{et al.} (~\citet{booji}) 
%%proposed navigation via image based topological map. 
%%The robot is driven through the environment and keeps on 
%%taking the pictures. The system uses epipolar constraints 
%%based on SIFT features to match the images. By conputing the 
%%similarities between these images a topological map 
%%in form of in form of an appearance graph is created.Navigation on 
%%this graph involves heading from one node to the other.
%%
%%Se \emph{et al.} (~\citet{se01}) proposed a vision based localisation and mapping 
%%algorithm which uses scale invariant image features as landmarks in unmodified 
%%dynamic environments. The 3D landmarks are localised and robot 
%%ego-motion is estimated by matching them 
%%taking into account the feature viewpoint extraction.
%%
%%Feature tracking navigation systems tracks features 
%%in consective frames to perform the navigation. 
%%Pears and Liang \emph{Et al.} (~\citet{liang02}) 
%%use hoographies to track ground plane corners 
%%in indoor buildings.  The same authors extended their 
%%work by using homographies to calculate the height of 
%%tracked features or obstacles above the ground plane during the navigation. 
%%
%%During robot driving throught the environment, SIFT based methods extract 
%%the important features from the environment which 
%%serve as landmarks to be tracked for navigation, global 
%%localisation and robust vision based SLAM performance 
%%(~\citet{se02, se05}).
%%%\begin{itemize}
%%%\item \textbf{Map based navigation:} 
%%%These systems require a map of the environment 
%%%for the navigation. It also includes those systems that can explore the environment 
%%%and build the map by themselves. However 
%%%the environment needs to be explored first 
%%%and its representation is stored before the start of navigation.
%%%Therefore systems in this category will start 
%%%navigation if and only the environment map 
%%%is available.
%%%
%%%\item \textbf{Mapless navigation:}
%%%It includes all navigation approaches that do not 
%%%require any knowledge of the environment for navigation
%%%run. Navigation is performed on the basis of elements 
%%%observed in the environment such as walls, features,
%%%doors, desks, etc. 
%%%
%%%\end{itemize}
%%%
%%%In the last few years, lot of progress is observed in context 
%%%of visual navigation. The older techniques have been refined 
%%%and lot of new techniques are also presented 
%%%which have led to a more accurate and 
%%%efficient navigation systems. We review some of the 
%%%visual navigation approaches used for map based and 
%%%mapless navigation in the following sections. 
%%%
%%%
%%%\subsection{Map based systems}
%%%This category includes systems that need a complete map 
%%%of the environment before the start of navigation.
%%%Other systems in this category are able to 
%%%explore the environment and 
%%%automatically build a map. However the navigation 
%%%phase will start only once the map is built. 
%%%The map information can be directly 
%%%used for navigation or it can be
%%%post-processed to improve the map accuracy
%%%to achieve a more precise localization. 
%%In the recent years, visual sonar is also used 
%%for the navigation purposes. It uses range data and depth measurements 
%%for navigation purposes in an analogous way to 
%%ultrasound sensors. Martin used the same idea 
%%to compute depth from single camera images of indoor 
%%environments (~\citet{martin06}).
%% The genetic programming is used to 
%%discover the best algorithm to detect the ground boundaries 
%%in a training phase. These algorithms then use obstacle 
%%avoidance strategies initially developed for sonar.
%%The algorithms are designed to be a 
%%replacement for sonar, returning the location of
%%the nearest obstacle in a given direction. The three 
%%indoor data sets were used with a total of 
%%images from four different hall ways.
%%

%%%
%%% 
%%%\section{Visual Indoor Navigation}
%%%\begin{enumerate}
%%%\item Map based navigation
%%%\begin{enumerate}
%%%\item Topological maps
%%%\item Local maps
%%%\end{enumerate}
%%%
%%%\item Map less navigation
%%%\begin{enumerate}
%%%\item Optical Flow
%%%\item Appearance based navigation
%%%\item Feature based navigation
%%%\end{enumerate}
%%%\end{enumerate}
%%
%%
%%%\section{Simultaneous Localisation and Mapping (SLAM)}
%%%A brief discussion about SLAM.
%%%\subsection{Visual SLAM}
%%%Discussion about SLAM based on images.
%%%I will mention some of the research works 
%%%based on Visual SLAM for robotics.
%%
%%\subsubsection{Loop Closure}
%%Discuss the importance 
%%of loop closure i.e. identification of the place 
%%which has been already visited.
%%
%%
%%
%%\section{Image based localization}
%%I will state that our indoor location recognition is 
%%also based on image based localisation.
%%We are doing a 2D image base matching.
%%I will discuss some research works briefly 
%%which have done localisation in indoor environments
%%using different techniques based on 2D images.
%
%
%\subsection{Indoor Navigation Systems}
%Simultaneous localization and mapping (SLAM) is a 
%technique used by robots and autonomous vehicles 
%to build up a map within an unknown environment (without a priori knowledge), or to update a map within a known environment (with a priori knowledge from a given map), while at the same time keeping track of their current location.
%
%Indoor navigation systems are based on different technologies 
%such as WIFI, laser range etc. Each technology used for 
%indoor navigation system has its pros and cons. Some commonly used 
%technologies for indoor navigation system are as follows:-
%
%%\subsection{Using Laser Scanner} 
%%In indoor, the localization using laser range 
%%finder has been used a lot. However these 
%%methods provide a 2D map of the environment. 
%%
%%The navigation system these days are based on different technologies.
%%Every technology has it pros and cons.
%%
%%\subsection{With WIFI Fingerprints}
%%\subsection{With Images}
%%\subsection{With Smartphones}
%
%
%%\section{Visual Navigation}
%%\label{sec:visualnavigation}
%%
%%The vision (image or videos) has become common in applications 
%%such as localisation, automatic map construction, 
%%autonomous navigation, path following etc in the 
%%last few years. The navigation techniques based 
%%on vision increase the scope of application of autonomous 
%%mobility and have resulted in countless research 
%%contributions for robots, autonomous ground vehicles, 
%%unmanned aerial vehicles and blind people. 
%%The systems that use vision for navigation can be 
%%divided into two main categories:-
%%
