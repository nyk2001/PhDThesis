\chapter{Smart phone application for scene localization}
A brief overview of the chapter. 

\section{System Implementation}
I will briefly discuss the client server architecture 
of our system. This will be followed by an general 
explanation about the working of our system 
i.e. photo taken by phone, compressed, sent to server etc.

\subsection{Mobile Operating System}
I will state different operating systems 
available. 
\begin{enumerate}
\item Android from Google
\item iOS from Apple
\item Window Phone from Microsoft
\item Symbian OS from Nokia
\end{enumerate}

I will then state the reason why I have used 
Android OS. 

\subsection{Mobile Phone Specifications}
I will mention some details of our HTC smart phone 
used to develop the application.

\section{Client Side}
I will discuss the key components of 
our smart phone application.
\begin{enumerate}
\item Waiting for Screen Tapping.
User can click any where, image will be captured.  
\item Image Compression.

Image is compressed by 50\% and sent to 
the server.

\item Waiting for reply.
Application waits for reply from the server.
Once it gets the reply then it generates 
a voice message indicating the current location.

\item Use of GPS
I will state the option of using GPS to load 
appropriate data set on the server.
\end{enumerate}

\subsection{Prototypes}
I will discuss our first prototype with couple of buttons and 
the final prototype with no button at all 
via screen shots.
\begin{enumerate}
\item First Prototype
\item Final Prototype
\end{enumerate}

\section{Server Side}
I will discuss the basic functionality of server i.e. image is received 
by servelet running on server which is copied to shared location on disk. 
Our localization program implemented in C++ 
reads the image and writes the location 
information on the shared location on disk. Once servelet finds the 
location it returns it to the android application. 


\subsection{Servelets}
I will briefly discuss the working of servelets i.e 
how they are set up, which tool I am using etc.

\subsection{Localization Program}
Here I will mention that our program is based on the 
same algorithm which was mentioned in the Chapter 4.
Homography and fundamental matrices are used 
for the verification. However the algorithm for real time 
application is extended 
to have an extra step of location validation via use of 
reduced features discussed in the previous 
chapter. I will then state the 
new algorithm:-

I will highlight that use of extra verification has almost no 
impact on matching accuracy but at the same it reduces the 
false acceptance rate significantly.  


\section{Datasets}
I will discuss the data sets which are used 
for testing our scene localization system.

\begin{enumerate}
\item Owheo Building
\item Commerce
\item Museum
\item Indoor
\item Outdoor
\end{enumerate}
\section{Performance Metrics}
\begin{enumerate}
\item True Positive rate
\item False acceptance rate
\item False positive rate
\end{enumerate}
\section{Results}
I will discuss that experiments are conducted 
in two ways:-

\begin{enumerate}
\item Single Data set:- All mapped images 
combined together.
\item Multiple Data sets:-- 
Depending upon the GPS coordinates, 
the appropriate data set is being loaded 
on the server.
\end{enumerate}

The performance metrics will be reported 
in both scenarios.

\subsection{True Positive Rate}
I will compare the true positive rate of 
our proposed BoW along with validation against:
\begin{itemize}
\item Simple Visual BoW : Top image is the best match
\item Proposed Visual BoW: Our proposed with verification method
only excluding the validation step. 
\end{itemize}

I will report the results on the data sets namely Owheo, 
Commerce and Museum.

\subsection{False Acceptance rate}
I will compare the false acceptance rate of 
our proposed BoW (verify+validation) against:
\begin{itemize}
\item Simple Visual BoW : Top image is the best match
\item Proposed Visual BoW: Our proposed with verification method
only. The validation step is not being used.
\end{itemize}

I will report the results on the data sets namely Owheo, 
Commerce and Museum. These results will show the 
improvement in false acceptance rate.


\subsection{False Positive rate}
I will compare the results on two data sets 
namely Indoor and Outdoor whose images 
are not there in the mapped images.

\subsection{Voting Scheme Analysis}
I will discuss the performance of our voting scheme 
in terms of real time scenario.

\subsection{Computational Time}
I will discuss the time taken by the homography and 
fundamental module to verify one image. 

What is the performance if we take all matching features 
and compute the fundamental matrix? I will discuss that as well. 

\section{Conclusion}